%
%  PACKAGE IMPORTS
%
\documentclass{article} % For LaTeX2e
\usepackage{iclr2022_conference,times}
\input{math_commands.tex}

\newcommand{\apsname}{Project Proposal}
\newcommand{\gpnumber}{26}

\usepackage{hyperref}
\usepackage{url}
\usepackage{graphicx}

%
%  TITLE AND AUTHORS
%
\title{Deep learning approach to  \\ 
mushroom species classification}

\author{Yanni Alan Alevras  \\
Student\# 1009330706 \\
\texttt{yanni.alevras@mail.utoronto.ca} \\
\And
Nicholas Biancolin  \\
Student\# 1005678901 \\
\texttt{n.biancolin@mail.utoronto.ca} \\
\AND
Eric Liu  \\
Student\# 1009098450 \\
\texttt{ey.liu@mail.utoronto.ca} \\
\And
Jason Ruixuan Zhang \\
Student\# 1008997631 \\
\texttt{jasonrx.zhang@mail.utoronto.ca} \\
\AND
}

\newcommand{\fix}{\marginpar{FIX}}
\newcommand{\new}{\marginpar{NEW}}

\iclrfinalcopy

%
%   DOCUMENT STARTS HERE
%
\begin{document}
\maketitle

\section{Introduction}
\label{intro}

The format for the submissions is a variant of the ICLR 2022 format.
Please read carefully the instructions below, and follow them
faithfully. There is a \textbf{9 page} limit for the main text. References do not have any limitation. This is also ICLR's standard length for a paper submission. 
If your main text goes to page 10, a $-20\%$ penalty would be applied. If your main text goes to page 11, you will not receive any grade for your submission. 

\section{Background \& Related Work}
\label{bg_related}

One key application of a fungi identification model is food safety and satisfaction.

In Bangladesh, a country with a large mushroom production, a farming method was developed by \cite{RahmanFaruqEtAl.IoTEnabledMushroom.2022} using machine learning to classify which mushrooms are being harvested. This system intended to remove toxic species that may have grown in the same area as the target mushroom. \cite{WangZhengEtAl.AutomaticSortingSystem.2018} created an algorithm to identify disease, discolouration, freshness, as well as other factors contributing to the commercial quality of a white button mushroom.

Our software would sort these mushrooms into genera, which would assist in identifying different types of mushrooms instead of certain physical features. Keeping with the theme of food quality, in Taiwan, \cite{LuLiawEtAl.DevelopmentMushroomGrowth.2019} developed a system to determine how much a mushroom has grown using an image recognition model, which produces an estimate based on images from different times. This used a convolutional neural network (CNN) to provide these results, which will be similar to our model's architecture. In the Chinese province of Yunnan, \cite{H.ZhaoF.GeEtAl.IdentificationWildMushroom.2021} created a wild mushroom identification model that used a CNN to identify edible and medicinal mushrooms due to increasing popular interest in mushrooms.

Other field work includes smartphone applications for recreational use, like ShroomID, which details mushroom species, while providing a heatmap of its location and seasonality. This educational tool provides useful information about a mushroom, after it has been identified using a classification model \citep{.ShroomID.2023}.

\section{Data Processing}
\label{data}

Many species in the dataset have a relatively small amount of images associated with them ($<30$ images), which may be detrimental when training, testing, and validating our model. To mitigate this, we intend to group images into larger “buckets” corresponding to a higher level of taxonomic classification, i.e., grouping by genus instead of by species. This is a simple way to reduce the number of classes our model must be trained on, and to increase the amount of data to a sufficient amount that model features can be reasonably trained. Genera form an ideal way to group images together — biologically species of the same genus share many physical characteristics, and they share a root name (the first word in the species name), which enhances ease of processing. Existing literature \citep{HollisterCaiEtAl.UsingComputerVision.2023} also suggests that models trained to classify genera tend to have better accuracy and are less difficult to train than species-level classification.

Once we have performed this combination, any genera with less than 75 images will be discarded from the dataset. This matches the lower threshold of published ecological classifiers, like iNaturalist \citep{Shepard.LatestComputerVision.2022} or models trained on the ETHZ human dataset \citep{SchneiderTaylorEtAl.PresentFutureApproaches.2019}. It also matches what is considered by \cite{FarleyMehrotaEtAl.ImprovingYourModel.2024} to be the lower end of images per class to train a model without overfitting.

\section{Citations, figures, tables, references}
\label{others}

These instructions apply to everyone, regardless of the formatter being used.

\subsection{Citations within the text}

Citations within the text should be based on the \texttt{natbib} package
and include the authors' last names and year (with the ``et~al.'' construct
for more than two authors). When the authors or the publication are
included in the sentence, the citation should not be in parenthesis using \verb|\citet{}| (as
in ``See \citet{Hinton06} for more information.''). Otherwise, the citation
should be in parenthesis using \verb|\citep{}| (as in ``Deep learning shows promise to make progress
towards AI~\citep{Bengio+chapter2007}.'').

The corresponding references are to be listed in alphabetical order of
authors, in the \textsc{References} section. As to the format of the
references themselves, any style is acceptable as long as it is used
consistently.

To cite a new paper, first, you need to add that paper's BibTeX information to \verb+APS360_ref.bib+ file and then you can use the \verb|\citep{}| command to cite that in your main document. 

\subsection{Footnotes}

Indicate footnotes with a number\footnote{Sample of the first footnote} in the
text. Place the footnotes at the bottom of the page on which they appear.
Precede the footnote with a horizontal rule of 2~inches
(12~picas).\footnote{Sample of the second footnote}

\subsection{Figures}

All artwork must be neat, clean, and legible. Lines should be dark
enough for purposes of reproduction; art work should not be
hand-drawn. The figure number and caption always appear after the
figure. Place one line space before the figure caption, and one line
space after the figure. The figure caption is lower case (except for
first word and proper nouns); figures are numbered consecutively.

Make sure the figure caption does not get separated from the figure.
Leave sufficient space to avoid splitting the figure and figure caption.

You may use color figures.
However, it is best for the
figure captions and the paper body to make sense if the paper is printed
either in black/white or in color.

\begin{figure}[h]
\begin{center}
\includegraphics[width=0.6\textwidth]{figs/td-deep-learning.jpg}
\end{center}
\caption{Sample figure caption. Image: ZDNet}
\end{figure}

\subsection{Tables}

All tables must be centered, neat, clean and legible. Do not use hand-drawn
tables. The table number and title always appear before the table. See
Table~\ref{sample-table}.

Place one line space before the table title, one line space after the table
title, and one line space after the table. The table title must be lower case
(except for first word and proper nouns); tables are numbered consecutively.

\begin{table}[t]
\caption{Sample table title}
\label{sample-table}
\begin{center}
\begin{tabular}{ll}
\multicolumn{1}{c}{\bf PART}  &\multicolumn{1}{c}{\bf DESCRIPTION}
\\ \hline \\
Dendrite         &Input terminal \\
Axon             &Output terminal \\
Soma             &Cell body (contains cell nucleus) \\
\end{tabular}
\end{center}
\end{table}

\section{Default Notation}

In an attempt to encourage standardized notation, we have included the
notation file from the textbook, \textit{Deep Learning}
\cite{goodfellow2016deep} available at
\url{https://github.com/goodfeli/dlbook_notation/}.  Use of this style
is not required and can be disabled by commenting out
\texttt{math\_commands.tex}.

\section{Final instructions}
Do not change any aspects of the formatting parameters in the style files.
In particular, do not modify the width or length of the rectangle the text
should fit into, and do not change font sizes (except perhaps in the
\textsc{References} section; see below). Please note that pages should be
numbered.


\subsubsection*{Author Contributions}
If you'd like to, you may include  a section for author contributions as is done
in many journals. This is optional and at the discretion of the authors.

\subsubsection*{Acknowledgments}
Use unnumbered third level headings for the acknowledgments. All
acknowledgments, including those to funding agencies, go at the end of the paper.

\label{last_page}

\bibliography{proposal_ref}
\bibliographystyle{iclr2022_conference}

\end{document}
