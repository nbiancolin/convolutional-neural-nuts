%
%  PACKAGE IMPORTS
%
\documentclass{article} % For LaTeX2e
\usepackage{iclr2022_conference,times}
\input{math_commands.tex}

\newcommand{\apsname}{Progress Report}
\newcommand{\gpnumber}{26}

\usepackage{hyperref}
\usepackage{url}
\usepackage{graphicx}
\usepackage[table]{xcolor}

\usepackage{graphicx} % Required for \resizebox
\usepackage{placeins} % Required for \FloatBarrier

%
%  TITLE AND AUTHORS
%
\title{Deep learning approach to  \\ 
mushroom species classification}

\author{Yanni Alan Alevras  \\
Student\# 1009330706 \\
\texttt{yanni.alevras@mail.utoronto.ca} \\
\And
Nicholas Biancolin  \\
Student\# 1009197726 \\
\texttt{n.biancolin@mail.utoronto.ca} \\
\AND
Eric Liu  \\
Student\# 1009098450 \\
\texttt{ey.liu@mail.utoronto.ca} \\
\And
Jason Ruixuan Zhang \\
Student\# 1008997631 \\
\texttt{jasonrx.zhang@mail.utoronto.ca} \\
\AND
}

\newcommand{\fix}{\marginpar{FIX}}
\newcommand{\new}{\marginpar{NEW}}

\iclrfinalcopy

%
%   DOCUMENT STARTS HERE
%

\begin{document}
\maketitle

\section{Project Description}
\label{sec:project_description}

- motivations behind project
    - 87% of Ontario's land is crown land (mostly forests), and it is filled with many species of plant
    - mushrooms are a common type of plant that are difficult to identify, especially since some are edible and some are poisonous

- goal of project
    - Create a deep learning model that can classify mushrooms into their respective species

- why deep learning is a reasonable approach
    - mushrooms have a lot of visual features that can be used to classify them
    - deep learning models have been shown to be effective at image classification tasks

\section{Indvidual Contributions and Responsibilities}
\label{sec:individual_contributions_and_responsibilities}

- How team is working together


-project management software used to communicate/track results


-detailed list of what everyone has worked on, and what they will be working on

Yanni:

For this milestone, Yanni contributed to the data augmentation portion of this project. Using 

\begin{center}
\begin{tabular}{ |c|c|c|c| }
\hline
\rowcolor{gray!50}
\textbf{Yanni} & \textbf{Nick} & \textbf{Eric} & \textbf{Jason} \\
\hline


\end{tabular}
\end{center}

\section{Notable Contribution}
\label{sec:notable_contribution}

Data Processing

The dataset contains # species. Stated in our project proposal, due to some low amount of sample imaging for some species we decided to group by genus. Any genus with under # samples would be removed. This narrowed us down from 509 species, and # genus, to 15 genus. To artificially increase the amount of data in our dataset, we used data augmentation, creating a copy of each sample with a horizontal flip, 90 degree rotation, 180 degree rotation, 270 degree rotation, gaussian noise, and random erasing (small black rectangles). These methods were preferred over others such as kernel filters, lowering the quality of an image. Since these mushrooms can be identified based on specific visual traits found within their genus, high detail in the training images are necessary to allow for differentiation between the different genera. Due to this, prioritizing the quality of the image was necessary. Some simple methods were the flip and rotations, which kept the same image, but just made the model look at it a different way. Gaussian noise was added .................... Random erasing was added to

Baseline Model


Primary Model




\end{document}
