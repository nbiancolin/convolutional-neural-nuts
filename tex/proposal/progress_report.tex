% !TEX root = project_proposal.tex
%
%  PACKAGE IMPORTS
%
\documentclass{article} % For LaTeX2e
\usepackage{iclr2022_conference,times}
\input{math_commands.tex}

\newcommand{\apsname}{Progress Report}
\newcommand{\gpnumber}{26}

\usepackage{hyperref}
\usepackage{url}
\usepackage{graphicx}
\usepackage[table]{xcolor}

\usepackage{graphicx} % Required for \resizebox
\usepackage{placeins} % Required for \FloatBarrier

%
%  TITLE AND AUTHORS
%
\title{Deep learning approach to  \\ 
mushroom species classification}

\author{Yanni Alan Alevras  \\
Student\# 1009330706 \\
\texttt{yanni.alevras@mail.utoronto.ca} \\
\And
Nicholas Biancolin  \\
Student\# 1009197726 \\
\texttt{n.biancolin@mail.utoronto.ca} \\
\AND
Eric Liu  \\
Student\# 1009098450 \\
\texttt{ey.liu@mail.utoronto.ca} \\
\And
Jason Ruixuan Zhang \\
Student\# 1008997631 \\
\texttt{jasonrx.zhang@mail.utoronto.ca} \\
\AND
}

\newcommand{\fix}{\marginpar{FIX}}
\newcommand{\new}{\marginpar{NEW}}

\iclrfinalcopy

%
%   DOCUMENT STARTS HERE
%

\begin{document}
\maketitle

\section{Project Description}
\label{sec:project_description}

- motivations behind project
    - 87% of Ontario's land is crown land (mostly forests), and it is filled with many species of plant
    - mushrooms are a common type of plant that are difficult to identify, especially since some are edible and some are poisonous

- goal of project
    - Create a deep learning model that can classify mushrooms into their respective species

- why deep learning is a reasonable approach
    - mushrooms have a lot of visual features that can be used to classify them
    - deep learning models have been shown to be effective at image classification tasks

\section{Indvidual Contributions and Responsibilities}
\label{sec:individual_contributions_and_responsibilities}

- How team is working together


-project management software used to communicate/track results


-detailed list of what everyone has worked on, and what they will be working on

Yanni:

For this milestone, Yanni contributed to the data augmentation portion of this project. Using 

\begin{center}
\begin{tabular}{ |c|c|c|c| }
\hline
\rowcolor{gray!50}
\textbf{Yanni} & \textbf{Nick} & \textbf{Eric} & \textbf{Jason} \\
\hline


\end{tabular}
\end{center}

\section{Notable Contribution}
\label{sec:notable_contribution}

Data Processing

The dataset contains # species. Stated in our project proposal, due to some low amount of sample imaging for some species we decided to group by genus. Any genus with under # samples would be removed. This narrowed us down from 509 species, and # genus, to 15 genus. To artificially increase the amount of data in our dataset, we used data augmentation, creating a copy of each sample with a horizontal flip, 90 degree rotation, 180 degree rotation, 270 degree rotation, gaussian noise, and random erasing (small black rectangles). These methods were preferred over others such as kernel filters, lowering the quality of an image. Since these mushrooms can be identified based on specific visual traits found within their genus, high detail in the training images are necessary to allow for differentiation between the different genera. Due to this, prioritizing the quality of the image was necessary. Some simple methods were the flip and rotations, which kept the same image, but just made the model look at it a different way. Gaussian noise was added .................... Random erasing was added to

Baseline Model


Primary Model
\begin{figure}[h]
    \begin{center}
    \includegraphics[width=0.6\textwidth]{figures/primaryModelDiagram.png}
    \end{center}
    \caption{Model Structure and Tensor Sizes}
\end{figure}


Our CNN model consists of two main sections: a non-tunable transfer learning part and a tunable convolution and fully connected layers section.

\section*{Non-Tunable Section}
  
The team uses AlexNet for its high-level feature extraction. AlexNet processes a $3 \times 244 \times 244$ input image and the feature extraction outputs a $256 \times 6 \times 6$ tensor \cite{10}. There are five convolutional layers and three pooling layers \cite{10}, the order of the layers is shown on the hand-drawn diagram above. Since the model needs to differentiate between mushrooms with very similar appearances, AlexNet excels in extracting the fine features that set them apart.
  
\section*{Tunable Section}
  
To make the model specific to the team’s project, the team uses one additional convolutional layer, outputting a $128 \times 6 \times 6$ tensor. After the additional convolutional layer, the output gets flattened and passed through three fully connected layers with ReLU activation functions in between. The fully connected layers turned the size from 4608 to 256, then to 128, and lastly to 30, matching the number of output classes the team decided for the model.
  
In total, there are $5+3$ layers in the non-tunable section and $1+3$ layers in the tunable section, making our class structure 12 layers in total.
  
\begin{figure}[h]
    \begin{center}
    \includegraphics[width=0.6\textwidth]{figures/AlexNetStructure.png}
    \end{center}
    \caption{Class Structure: AlexNet \cite{10}}
\end{figure}

\section*{Calculation of Parameters}

\subsection*{Number of parameters for the AlexNet structure:}
\begin{align*}
\text{Conv1} & = 3 \times 11 \times 11 \times (96 + 1) = 35,271 \\
\text{Conv2} & = 96 \times 5 \times 5 \times (256 + 1) = 616,800 \\
\text{Conv3} & = 256 \times 3 \times 3 \times (384 + 1) = 886,080 \\
\text{Conv4} & = 384 \times 3 \times 3 \times (384 + 1) = 1,310,720 \\
\text{Conv5} & = 384 \times 3 \times 3 \times (256 + 1) = 887,232 \\
\end{align*}

\subsection*{Number of parameters for the tunable section:}
\begin{align*}
\text{Conv1} & = 256 \times 3 \times 3 \times (128 + 1) = 297,216 \\
\text{Fc1} & = 4608 \times (256 + 1) = 1,183,296 \\
\text{Fc2} & = 256 \times (128 + 1) = 33,024 \\
\text{Fc3} & = 128 \times (30 + 1) = 3,968 \\
\end{align*}

The total number of parameters is 5,273,927, the number of trainable parameters is only 1,517,504. This ensures the training time for our models is feasible, allowing the team to focus on more epochs and more variations using data augmentations in the future.

At the start, the team pushed all images into the feature extraction part of AlexNet, converting data into tensors. We randomly split the data into a 75\%, 15\%, and 10\% ratio for training, validation, and testing.

For our current best result, we used a batch size of 36, learning rate of 0.007, and 15 epochs. We chose Cross Entropy Loss for the loss function as we want the model to classify the image into one of the 30 classes. For the optimizer, the group decided on Stochastic Gradient Descent (SGD).


\end{document}
